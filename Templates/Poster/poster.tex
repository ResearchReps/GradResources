\documentclass{beamer}

\usepackage[orientation=portrait,size=a0,scale=1.1]{beamerposter}
\usetheme{confposter}
\usepackage{exscale} % Use after theme to display integrals properly

\usepackage{amsmath,amssymb,amsthm}
\usepackage{stmaryrd}
\usepackage{enumerate}
\usepackage{stfloats}
\usepackage{flushend}

\usepackage{tikz}
\usetikzlibrary{arrows,shapes,chains,matrix,positioning,scopes}
\usepackage{pgfplots}
\usepgflibrary{shapes}
\pgfplotsset{compat=1.8}

% \newtheoremstyle{custom}
% {} % Space above
% {} % Space below
% {} % Body font
% {} % Indent amount
% {\bfseries} % Theorem head font
% {:} % Punctuation after theorem head
% {.25em} % Space after theorem head
% {} % Theorem head spec (can be left empty, meaning `normal')
% \theoremstyle{custom}

% \newtheorem{theorem}{Theorem}
% \newtheorem{lemma}[theorem]{Lemma}
% \newtheorem{proposition}[theorem]{Proposition}
% \newtheorem{definition}[theorem]{Definition}
% \newtheorem{example}[theorem]{Example}
% \newtheorem{remark}[theorem]{Remark}
% \newtheorem{corollary}[theorem]{Corollary}

% \newtheorem*{theorem*}{Theorem}
% \newtheorem*{lemma*}{Lemma}
% \newtheorem*{proposition*}{Proposition}
% \newtheorem*{definition*}{Definition}
% \newtheorem*{example*}{Example}
% \newtheorem*{remark*}{Remark}
% \newtheorem*{corollary*}{Corollary}

\renewcommand{\epsilon}{\varepsilon}

\newcommand{\h}{\texttt{h}}
\newcommand{\hbp}{\h^{\mathrm{BP}}}
\newcommand{\hmap}{\h^{\mathrm{MAP}}}
\newcommand{\hstab}{\h^{\mathrm{stab}}}
\newcommand{\harea}{\h^{A}}

\global\long\def\expt{\mathbb{E}}
\newcommand{\indicator}[1]{\mathbbm{1}_{\left\{ {#1} \right\} }}

\newcommand{\mb}[1]{\mathbf{#1}}
\newcommand{\mbb}[1]{\mathbb{#1}}
\newcommand{\mr}[1]{\mathrm{#1}}
\newcommand{\mc}[1]{\mathcal{#1}}
\newcommand{\ms}[1]{\mathsf{#1}}
\newcommand{\mf}[1]{\mathfrak{#1}}

\newcommand{\mse}{\mathsf{e}}
\newcommand{\msx}{\mathsf{x}}
\newcommand{\msxvn}{\tilde{\mathsf{x}}}
\newcommand{\msy}{\mathsf{y}}
\newcommand{\msz}{\mathsf{z}}
\newcommand{\msa}{\mathsf{a}}
\newcommand{\msb}{\mathsf{b}}
\newcommand{\msc}{\mathsf{c}}
\newcommand{\msbx}{\underline{\mathsf{x}}}
\newcommand{\msby}{\underline{\mathsf{y}}}
\newcommand{\msbxvn}{\tilde{\underline{\mathsf{x}}}}
\newcommand{\msbz}{\underline{\mathsf{z}}}
\newcommand{\msba}{\underline{\mathsf{a}}}
\newcommand{\msbb}{\underline{\mathsf{b}}}
\newcommand{\msbc}{\underline{\mathsf{c}}}

\newcommand{\bop}{\ast}
\newcommand{\vnop}{\varoast}
\newcommand{\disth}{d_{\mathrm{H}}}
\newcommand{\cnop}{\boxast}
\newcommand{\diff}[1]{d#1}
\newcommand{\deri}[1]{\mathrm{d}_{ #1 }\hspace{0.05cm}}
\newcommand{\dderi}[1]{\mathrm{d}_{ #1 }^2\hspace{0.05cm}}
\newcommand{\bvert}[1]{\,\Big{\vert}_{ #1  }}
\newcommand{\degr}{\succ}
\newcommand{\degreq}{\succeq}
\newcommand{\upgr}{\prec}
\newcommand{\upgreq}{\preceq}
\newcommand{\extR}{\overline{\mathbb{R}}}

\newcommand{\des}{\mathsf{T}_\mathrm{s}}
\newcommand{\dec}{\mathsf{T}_\mathrm{c}}
\newcommand{\pots}{U_\mathrm{s}}
\newcommand{\potc}{U_\mathrm{c}}
\newcommand{\shft}{\mathsf{S}}

\newcommand{\vnunit}{\Delta_0}
\newcommand{\cnunit}{\Delta_\infty}

\newcommand{\ent}[1]{ \mathrm{H} \left( #1 \right) }

\newcommand{\meass}{\mathcal{M}}
\newcommand{\probs}{\mathcal{X}}
\newcommand{\dpros}{\mathcal{X}_{\mathrm{d}}}

\newcommand{\chend}{N_{w}}

\newcommand{\minf}{\mathsf{a}_{0}}
\newcommand{\minfb}{\underline{\minf}}

\DeclareMathOperator*{\argmin}{\,arg\ min}
\DeclareMathOperator*{\argmax}{\,arg\ max}
\providecommand{\abs}[1]{\left\lvert#1\right\rvert}

\newlength\tikzwidth
\newlength\tikzheight

\textfloatsep=0.05in

%%% Local Variables: 
%%% mode: plain-tex
%%% TeX-master: "journal"
%%% End: 



\newlength{\columnheight}
\newlength{\onecolwid}
\newlength{\twocolwid}
\newlength{\threecolwid}
\newlength{\halfcolwid}
\setlength{\columnheight}{105cm}
\setlength{\onecolwid}{0.3\paperwidth}
\setlength{\twocolwid}{0.3\paperwidth}
\setlength{\threecolwid}{0.3\paperwidth}
\setlength{\halfcolwid}{0.12\paperwidth}

\setbeamercolor{block title}{fg=ngreen,bg=white}
\setbeamercolor{block body}{fg=black,bg=white}
\setbeamercolor{block alerted title}{fg=white,bg=dblue!70}
\setbeamercolor{block alerted body}{fg=black,bg=dblue!10}

\title{\LARGE A Proof of Threshold Saturation for Spatially-Coupled Codes on BMS Channels}
\author{Santhosh Kumar$^{\dagger}$, Andrew J. Young$^{\ddagger}$, Nicolas Macris$^{\S}$, Henry D. Pfister$^{\star}$}
\institute{Texas A\&M University$^{\dagger}$, Massachusetts Institute of Technology$^{\ddagger}$, \'{E}cole Polytechnique F\'{e}d\'{e}rale de Lausanne$^{\S}$, Duke University$^\star$}

\begin{document}
\begin{columns}[t]
  % First Column
  \begin{column}{\onecolwid}

    \vspace{2cm}  
    % Introduction
    \begin{block}{\Large Introduction}
      \begin{itemize}
      \item \vspace{0.75cm} In 1999, Felstrom and Zigangirov introduced LDPC convolutional codes
      \item \vspace{0.75cm} In 2005, LSZC showed that terminated regular LDPC convolutional codes have BP thresholds close to capacity
      \item \vspace{0.75cm} KRU recently showed that this is a general phenomenon by establishing that spatially-coupled regular LDPC codes universally achieve capacity over BMS channels
      \item \vspace{0.75cm} YJNP gave a simple proof of threshold saturation for coupled scalar recursions (e.g., irregular LDPC over BEC) based on potential functions
      \item \vspace{0.75cm} We extend the proof technique based on potential functions to BMS channels
      \end{itemize}
    \end{block}

    % Spatially-coupled Ensemble
    \vspace{2cm}
    \begin{block}{\Large Spatially-Coupled Ensemble}
      \resizebox{26cm}{17cm}{
        \begin{tikzpicture}
  [
  node distance = 12mm, draw=black, thick, >=stealth',
  bitnode/.style={circle, inner sep = 0pt, minimum size = 4mm, draw=black},
  bitnode2/.style={circle, inner sep = 0pt, minimum size = 4mm, draw=black, fill=gray},
  checknode/.style={rectangle, inner sep = 0pt, minimum size = 4mm, draw=black},
  ]

  \draw[<->] (3,0.5) -- node[midway,fill=white]{\tiny{$N$}} (6,0.5);
  \draw[<->] (1,0.5) -- node[midway,fill=white]{\tiny{$w$}} (3,0.5);

  \foreach \x in {1,2} {
    \node[bitnode2] (bu\x) at (\x,0) {};
  }

  \foreach \x in {3,4,...,6} {
    \node[bitnode] (bu\x) at (\x,0) {};
  }

  \foreach \x in {7,8} {
    \node[bitnode2] (bu\x) at (\x,0) {};
  }
  
  \foreach \x in {1,2,...,6} {
    \node[checknode] (c\x) at (\x+2,-2) {};
  }

  \foreach \x in {1,2} {
    \node[bitnode2] (bd\x) at (\x,-4) {};
  }

  \foreach \x in {3,4,...,6} {
    \node[bitnode] (bd\x) at (\x,-4) {};
  }

  \foreach \x in {7,8} {
    \node[bitnode2] (bd\x) at (\x,-4) {};
  }

  \draw[dashed] (bu1) -- (c1); \draw[dashed] (bd1) -- (c1);
  \draw[dashed] (bu2) -- (c1); \draw[dashed] (bd2) -- (c1);
  \draw[] (bu3) -- (c1); \draw[] (bd3) -- (c1);

  \draw[dashed] (bu2) -- (c2); \draw[dashed] (bd2) -- (c2);
  \draw[] (bu3) -- (c2); \draw[] (bd3) -- (c2);
  \draw[] (bu4) -- (c2); \draw[] (bd4) -- (c2);

  \draw[] (bu3) -- (c3); \draw[] (bd3) -- (c3);
  \draw[] (bu4) -- (c3); \draw[] (bd4) -- (c3);
  \draw[] (bu5) -- (c3); \draw[] (bd5) -- (c3);

  \draw[] (bu4) -- (c4); \draw[] (bd4) -- (c4);
  \draw[] (bu5) -- (c4); \draw[] (bd5) -- (c4);
  \draw[] (bu6) -- (c4); \draw[] (bd6) -- (c4);

  \draw[] (bu5) -- (c5); \draw[] (bd5) -- (c5);
  \draw[] (bu6) -- (c5); \draw[] (bd6) -- (c5);
  \draw[dashed] (bu7) -- (c5); \draw[dashed] (bd7) -- (c5);

  \draw[] (bu6) -- (c6); \draw[] (bd6) -- (c6);
  \draw[dashed] (bu7) -- (c6); \draw[dashed] (bd7) -- (c6);
  \draw[dashed] (bu8) -- (c6); \draw[dashed] (bd8) -- (c6);

\end{tikzpicture}

%%% Local Variables: 
%%% mode: latex
%%% TeX-master: "../poster"
%%% End: 

      }
      \begin{itemize}
      \item \vspace{0.75cm} The protograph of a $(3,6)$ SC LDPC ensemble
      \item \vspace{0.75cm} The blue nodes represent the perfect information at the boundary
      \end{itemize}
    \end{block}

    % Main Theorems
    \vspace{2cm}
    \begin{alertblock}{\Large Main Result}
      \begin{itemize}
      \item Consider a SC irregular LDPC ensemble and ordered BMS channels $\msc(\h)$ with entropy $\h$ 
        \begin{itemize}
        \item \vspace{0.75cm} If $\h < \hmap$, then for large $w$, SC Density Evolution (DE) converges to the perfect decoding solution
        \item \vspace{0.75cm} If $\h > \hmap$, then for fixed $w$, and large $N$, the SC DE does not converge to the perfect decoding solution
        \end{itemize}
        \vspace{1.5cm}
      \item Consider a SC irregular LDGM ensemble and a BMS channel
        \begin{itemize}
        \item \vspace{0.75cm} When the DE of LDGM ensembles is initialized with perfect information, it converges to a nontrivial \textcolor{red}{minimal fixed point}
        \item \vspace{0.75cm} If the so-called \textcolor{red}{energy gap} for the BMS channel is positive, then the SC DE converges to a fixed point which is elementwise better than minimal fixed point
        \end{itemize}
      \end{itemize}
    \end{alertblock}

    % Preliminaries & Notation
    \vspace{2cm}
    \begin{block}{\Large Preliminaries}
      \begin{itemize}
      \item\vspace{0.75cm} $\msx$ denotes a \textcolor{red}{symmetric} probability measures on $\extR$
      \item\vspace{0.75cm} Entropy functional is given by $\ent{\msx} \triangleq \int \left( 1+e^{-\alpha} \right) \msx(\diff{\alpha})$
        \vspace{0.75cm}
        \begin{itemize}
        \item The \textcolor{red}{duality rule} states
          \begin{align*}
            \ent{\msx_1} + \ent{\msx_2} &= \ent{\msx_1 \vnop \msx_2} + \ent{\msx_1 \cnop \msx_2}
          \end{align*}
          \begin{align*}
            \ent{(\msx_1 - \msx_2) \vnop (\msx_3 - \msx_4) } + \ent{(\msx_1 - \msx_2) \cnop (\msx_3 - \msx_4) } = 0
          \end{align*}
        \end{itemize}
        
      \item\vspace{0.75cm} The variable-node and check-node operators are denoted by $\vnop$ and $\cnop$, respectively
        \begin{itemize}
          \vspace{0.5cm} \item Associative, commutative, linear
          \vspace{0.5cm} \item Preserves degradation order
        \end{itemize}
        
      \item\vspace{0.75cm} The \textcolor{red}{directional derivative} of a functional $F$ is in the direction $\msy$
        \begin{align*}
          \deri{\msx} F(\msx)[\msy] \triangleq \lim_{\delta \rightarrow 0} \frac{F(\msx + \delta \msy ) - F(\msx)}{\delta} 
        \end{align*}
        
      \item\vspace{0.75cm} If $F$ is a linear functional, the properties of $\vnop$ and $\cnop$ allow
        \begin{align*}
          \deri{\msx} F(\msx^{\vnop n})[\msy] &= n F( \msx^{\vnop (n-1)} \vnop \msy ) ,
        \end{align*}
      \end{itemize}
    \end{block}
  \end{column}

  % Second Column
  \begin{column}{\twocolwid}
    \vspace{2cm}
    \begin{block}{Potential Functional for LDPC Codes}
      \vspace{0.75cm}
      \setlength\tikzheight{14.5cm}
      \setlength\tikzwidth{19cm} 
      % This file was created by matlab2tikz v0.1.2.
% Copyright (c) 2008--2011, Nico Schlömer <nico.schloemer@gmail.com>
% All rights reserved.
% 
% The latest updates can be retrieved from
%   http://www.mathworks.com/matlabcentral/fileexchange/22022-matlab2tikz
% where you can also make suggestions and rate matlab2tikz.
% 
\begin{tikzpicture}

\tikzstyle{every node}=[font=\small]

\definecolor{gray1}{rgb}{0.2,0.2,0.2}
\definecolor{mycolor1}{rgb}{0,0.5,0}
\definecolor{mycolor2}{rgb}{0,0.75,0.75}
\definecolor{mycolor3}{rgb}{0.75,0,0.75}

\begin{axis}[%
scale only axis,
width=\tikzwidth,
height=\tikzheight,
xmin=0, xmax=1,
ymin=-0.02, ymax=0.1,
xlabel={\Large{$\tilde{\h}$}},
ylabel={\Large{$\pots\left( \mathrm{BAWGNC}(\tilde{\h}) ; \mathrm{BSC}(\h) \right)$}},
xtick={0,0.2,0.4,0.6,0.8,1},
ytick={0,0.02,0.04,0.06,0.08,0.10},
yticklabels={0,0.02,0.04,0.06,0.08,0.10},
xmajorgrids,
ymajorgrids,
zmajorgrids,
]


\addplot [
color=blue,
solid,
line width=3pt,
]
coordinates{
 (2.22045e-16,-2.39524e-09)(0.00408163,0.000175428)(0.00816327,0.000621376)(0.0122449,0.00126786)(0.0163265,0.00206782)(0.0204082,0.00298508)(0.0244898,0.00399121)(0.0285714,0.00506311)(0.0326531,0.00618122)(0.0367347,0.0073304)(0.0408163,0.00849582)(0.044898,0.00966678)(0.0489796,0.010834)(0.0530612,0.0119898)(0.0571429,0.0131271)(0.0612245,0.0142406)(0.0653061,0.0153263)(0.0693878,0.0163793)(0.0734694,0.0173984)(0.077551,0.018381)(0.0816327,0.019325)(0.0857143,0.0202296)(0.0897959,0.0210941)(0.0938776,0.021918)(0.0979592,0.0227014)(0.102041,0.0234444)(0.106122,0.0241477)(0.110204,0.0248122)(0.114286,0.0254388)(0.118367,0.0260285)(0.122449,0.0265825)(0.126531,0.0271025)(0.130612,0.0275895)(0.134694,0.0280454)(0.138776,0.0284716)(0.142857,0.0288699)(0.146939,0.0292418)(0.15102,0.0295893)(0.155102,0.0299138)(0.159184,0.030217)(0.163265,0.030501)(0.167347,0.0307671)(0.171429,0.0310174)(0.17551,0.0312532)(0.179592,0.0314764)(0.183673,0.0316884)(0.187755,0.031891)(0.191837,0.0320856)(0.195918,0.0322736)(0.2,0.0324566)(0.2,0.0324566)(0.216327,0.033165)(0.232653,0.0338944)(0.24898,0.0347118)(0.265306,0.0356696)(0.281633,0.0368066)(0.297959,0.0381487)(0.314286,0.0397097)(0.330612,0.0414934)(0.346939,0.0434941)(0.363265,0.0456989)(0.379592,0.0480886)(0.395918,0.0506393)(0.412245,0.0533237)(0.428571,0.0561119)(0.444898,0.0589736)(0.461224,0.0618777)(0.477551,0.0647935)(0.493878,0.0676918)(0.510204,0.0705451)(0.526531,0.0733284)(0.542857,0.0760187)(0.559184,0.0785964)(0.57551,0.0810443)(0.591837,0.0833489)(0.608163,0.0854995)(0.62449,0.087488)(0.640816,0.0893093)(0.657143,0.0909622)(0.673469,0.0924463)(0.689796,0.093765)(0.706122,0.0949232)(0.722449,0.0959275)(0.738776,0.096787)(0.755102,0.0975113)(0.771429,0.0981115)(0.787755,0.0985995)(0.804082,0.0989876)(0.820408,0.0992885)(0.836735,0.0995144)(0.853061,0.0996777)(0.869388,0.0997898)(0.885714,0.0998614)(0.902041,0.0999025)(0.918367,0.0999221)(0.934694,0.0999288)(0.95102,0.0999297)(0.967347,0.0999295)(0.983673,0.0999295)(1,0.0999295) 
};

\addplot [
color=red,
solid,
line width=3pt,
]
coordinates{
 (2.22045e-16,-2.36491e-09)(0.00408163,0.000175142)(0.00816327,0.000619424)(0.0122449,0.00126197)(0.0163265,0.00205507)(0.0204082,0.00296205)(0.0244898,0.0039541)(0.0285714,0.00500778)(0.0326531,0.00610332)(0.0367347,0.00722537)(0.0408163,0.00835899)(0.044898,0.00949336)(0.0489796,0.0106192)(0.0530612,0.0117286)(0.0571429,0.0128146)(0.0612245,0.013872)(0.0653061,0.0148967)(0.0693878,0.0158838)(0.0734694,0.0168323)(0.077551,0.0177395)(0.0816327,0.0186035)(0.0857143,0.0194236)(0.0897959,0.020199)(0.0938776,0.0209294)(0.0979592,0.0216152)(0.102041,0.0222564)(0.106122,0.0228538)(0.110204,0.0234086)(0.114286,0.0239216)(0.118367,0.0243942)(0.122449,0.0248276)(0.126531,0.0252234)(0.130612,0.0255832)(0.134694,0.0259086)(0.138776,0.0262014)(0.142857,0.0264634)(0.146939,0.0266964)(0.15102,0.0269022)(0.155102,0.0270828)(0.159184,0.0272398)(0.163265,0.0273753)(0.167347,0.027491)(0.171429,0.0275889)(0.17551,0.0276706)(0.179592,0.027738)(0.183673,0.0277927)(0.187755,0.0278366)(0.191837,0.0278712)(0.195918,0.0278981)(0.2,0.027919)(0.2,0.027919)(0.216327,0.0279711)(0.232653,0.028037)(0.24898,0.02819)(0.265306,0.0284888)(0.281633,0.0289774)(0.297959,0.0296862)(0.314286,0.0306331)(0.330612,0.0318252)(0.346939,0.0332595)(0.363265,0.0349253)(0.379592,0.036805)(0.395918,0.0388759)(0.412245,0.0411111)(0.428571,0.0434814)(0.444898,0.045956)(0.461224,0.0485034)(0.477551,0.0510925)(0.493878,0.0536928)(0.510204,0.0562758)(0.526531,0.0588153)(0.542857,0.061287)(0.559184,0.0636695)(0.57551,0.0659444)(0.591837,0.0680964)(0.608163,0.0701132)(0.62449,0.0719854)(0.640816,0.0737062)(0.657143,0.0752729)(0.673469,0.0766838)(0.689796,0.0779408)(0.706122,0.0790474)(0.722449,0.0800093)(0.738776,0.0808342)(0.755102,0.0815307)(0.771429,0.082109)(0.787755,0.08258)(0.804082,0.0829552)(0.820408,0.0832466)(0.836735,0.0834657)(0.853061,0.0836243)(0.869388,0.0837334)(0.885714,0.0838032)(0.902041,0.0838434)(0.918367,0.0838626)(0.934694,0.0838691)(0.95102,0.08387)(0.967347,0.0838698)(0.983673,0.0838698)(1,0.0838698) 
};

\addplot [
color=mycolor1,
solid,
line width=3pt,
]
coordinates{
 (2.22045e-16,-2.39432e-09)(0.00408163,0.000174721)(0.00816327,0.000616544)(0.0122449,0.00125328)(0.0163265,0.00203625)(0.0204082,0.00292806)(0.0244898,0.0038993)(0.0285714,0.0049261)(0.0326531,0.00598831)(0.0367347,0.00707029)(0.0408163,0.00815694)(0.044898,0.00923728)(0.0489796,0.0103019)(0.0530612,0.0113428)(0.0571429,0.0123531)(0.0612245,0.0133275)(0.0653061,0.014262)(0.0693878,0.0151518)(0.0734694,0.0159959)(0.077551,0.0167916)(0.0816327,0.0175373)(0.0857143,0.0182322)(0.0897959,0.0188759)(0.0938776,0.0194682)(0.0979592,0.0200094)(0.102041,0.0204999)(0.106122,0.0209408)(0.110204,0.0213331)(0.114286,0.0216781)(0.118367,0.0219772)(0.122449,0.0222319)(0.126531,0.022444)(0.130612,0.0226151)(0.134694,0.0227473)(0.138776,0.0228425)(0.142857,0.0229025)(0.146939,0.0229296)(0.15102,0.0229256)(0.155102,0.0228926)(0.159184,0.0228328)(0.163265,0.0227481)(0.167347,0.0226407)(0.171429,0.0225124)(0.17551,0.0223654)(0.179592,0.0222016)(0.183673,0.0220228)(0.187755,0.021831)(0.191837,0.0216281)(0.195918,0.0214158)(0.2,0.0211958)(0.2,0.0211958)(0.216327,0.0202724)(0.232653,0.0193511)(0.24898,0.0185151)(0.265306,0.0178318)(0.281633,0.0173531)(0.297959,0.0171165)(0.314286,0.0171457)(0.330612,0.0174528)(0.346939,0.018039)(0.363265,0.0188969)(0.379592,0.0200115)(0.395918,0.0213617)(0.412245,0.0229218)(0.428571,0.024663)(0.444898,0.0265545)(0.461224,0.0285643)(0.477551,0.0306602)(0.493878,0.0328105)(0.510204,0.034985)(0.526531,0.0371557)(0.542857,0.0392962)(0.559184,0.041383)(0.57551,0.0433955)(0.591837,0.045316)(0.608163,0.0471298)(0.62449,0.0488253)(0.640816,0.0503934)(0.657143,0.051829)(0.673469,0.0531284)(0.689796,0.0542914)(0.706122,0.0553197)(0.722449,0.0562169)(0.738776,0.056989)(0.755102,0.0576432)(0.771429,0.0581879)(0.787755,0.0586329)(0.804082,0.0589883)(0.820408,0.0592651)(0.836735,0.0594737)(0.853061,0.0596251)(0.869388,0.0597294)(0.885714,0.0597964)(0.902041,0.0598349)(0.918367,0.0598533)(0.934694,0.0598596)(0.95102,0.0598604)(0.967347,0.0598603)(0.983673,0.0598603)(1,0.0598603) 
};

\addplot [
color=mycolor2,
solid,
line width=3pt,
]
coordinates{
 (2.22045e-16,-2.28226e-09)(0.00408163,0.000174222)(0.00816327,0.000613135)(0.0122449,0.00124299)(0.0163265,0.00201397)(0.0204082,0.00288781)(0.0244898,0.0038344)(0.0285714,0.00482932)(0.0326531,0.00585204)(0.0367347,0.00688651)(0.0408163,0.00791746)(0.044898,0.0089337)(0.0489796,0.00992565)(0.0530612,0.0108854)(0.0571429,0.0118058)(0.0612245,0.0126817)(0.0653061,0.0135091)(0.0693878,0.0142834)(0.0734694,0.0150034)(0.077551,0.0156668)(0.0816327,0.0162719)(0.0857143,0.0168182)(0.0897959,0.0173053)(0.0938776,0.0177333)(0.0979592,0.0181027)(0.102041,0.0184141)(0.106122,0.0186688)(0.110204,0.0188679)(0.114286,0.0190129)(0.118367,0.0191056)(0.122449,0.0191476)(0.126531,0.0191409)(0.130612,0.0190876)(0.134694,0.0189897)(0.138776,0.0188494)(0.142857,0.0186689)(0.146939,0.0184505)(0.15102,0.0181964)(0.155102,0.017909)(0.159184,0.0175907)(0.163265,0.0172434)(0.167347,0.0168698)(0.171429,0.0164718)(0.17551,0.0160519)(0.179592,0.0156121)(0.183673,0.0151546)(0.187755,0.0146814)(0.191837,0.0141948)(0.195918,0.0136968)(0.2,0.013189)(0.2,0.013189)(0.216327,0.0110995)(0.232653,0.00899705)(0.24898,0.00697654)(0.265306,0.005116)(0.281633,0.00347692)(0.297959,0.00210468)(0.314286,0.00103066)(0.330612,0.00027278)(0.346939,-0.000162488)(0.363265,-0.000278671)(0.379592,-8.76637e-05)(0.395918,0.000391798)(0.412245,0.00113553)(0.428571,0.00211534)(0.444898,0.00330041)(0.461224,0.00465816)(0.477551,0.00615539)(0.493878,0.00775899)(0.510204,0.00943694)(0.526531,0.0111589)(0.542857,0.0128961)(0.559184,0.0146227)(0.57551,0.0163152)(0.591837,0.0179532)(0.608163,0.0195193)(0.62449,0.0209989)(0.640816,0.0223804)(0.657143,0.0236558)(0.673469,0.024819)(0.689796,0.0258672)(0.706122,0.0267996)(0.722449,0.0276177)(0.738776,0.0283255)(0.755102,0.0289279)(0.771429,0.0294318)(0.787755,0.0298451)(0.804082,0.0301765)(0.820408,0.0304356)(0.836735,0.0306316)(0.853061,0.0307744)(0.869388,0.0308732)(0.885714,0.0309369)(0.902041,0.0309737)(0.918367,0.0309915)(0.934694,0.0309976)(0.95102,0.0309984)(0.967347,0.0309983)(0.983673,0.0309983)(1,0.0309983) 
};

\addplot [
color=mycolor3,
solid,
line width=3pt,
]
coordinates{
 (2.22045e-16,-2.36461e-09)(0.00408163,0.000174033)(0.00816327,0.000611843)(0.0122449,0.00123909)(0.0163265,0.00200553)(0.0204082,0.00287256)(0.0244898,0.00380982)(0.0285714,0.00479268)(0.0326531,0.00580043)(0.0367347,0.00681692)(0.0408163,0.00782678)(0.044898,0.00881875)(0.0489796,0.0097832)(0.0530612,0.0107122)(0.0571429,0.0115986)(0.0612245,0.0124372)(0.0653061,0.013224)(0.0693878,0.0139546)(0.0734694,0.0146276)(0.077551,0.0152408)(0.0816327,0.0157926)(0.0857143,0.0162826)(0.0897959,0.0167104)(0.0938776,0.0170762)(0.0979592,0.0173805)(0.102041,0.017624)(0.106122,0.017808)(0.110204,0.0179339)(0.114286,0.0180032)(0.118367,0.0180176)(0.122449,0.017979)(0.126531,0.0178893)(0.130612,0.0177509)(0.134694,0.0175657)(0.138776,0.0173361)(0.142857,0.0170644)(0.146939,0.0167528)(0.15102,0.0164038)(0.155102,0.0160199)(0.159184,0.0156034)(0.163265,0.0151564)(0.167347,0.0146818)(0.171429,0.0141814)(0.17551,0.0136578)(0.179592,0.0131132)(0.183673,0.0125498)(0.187755,0.0119698)(0.191837,0.0113754)(0.195918,0.0107687)(0.2,0.0101515)(0.2,0.0101515)(0.216327,0.00761859)(0.232653,0.00506664)(0.24898,0.00259501)(0.265306,0.000285744)(0.281633,-0.00179604)(0.297959,-0.00360188)(0.314286,-0.00509757)(0.330612,-0.00626287)(0.346939,-0.00708934)(0.363265,-0.00757894)(0.379592,-0.00774233)(0.395918,-0.00759735)(0.412245,-0.00716758)(0.428571,-0.00648088)(0.444898,-0.00556805)(0.461224,-0.00446181)(0.477551,-0.00319573)(0.493878,-0.00180343)(0.510204,-0.000317618)(0.526531,0.00123055)(0.542857,0.00281154)(0.559184,0.00439837)(0.57551,0.00596664)(0.591837,0.00749489)(0.608163,0.00896463)(0.62449,0.0103603)(0.640816,0.0116691)(0.657143,0.0128821)(0.673469,0.0139922)(0.689796,0.0149955)(0.706122,0.0158904)(0.722449,0.0166776)(0.738776,0.01736)(0.755102,0.0179421)(0.771429,0.0184299)(0.787755,0.0188306)(0.804082,0.0191525)(0.820408,0.0194044)(0.836735,0.0195953)(0.853061,0.0197344)(0.869388,0.0198308)(0.885714,0.0198929)(0.902041,0.0199289)(0.918367,0.0199462)(0.934694,0.019952)(0.95102,0.0199528)(0.967347,0.0199526)(0.983673,0.0199526)(1,0.0199526) 
};

\draw[<->,very thick, gray1] (axis description cs:0.3,0.165) -- node[right=0.1pt]{\Large{$\Delta E$}} (axis description cs:0.3,0.31);
\node[fill=white] at (axis description cs:0.90,0.87) {\Large{$\h^{\mathrm{BP}}$}};
\node[fill=white] at (axis description cs:0.90,0.43) {\Large{$\h^{\mathrm{MAP}}$}};

\end{axis}
\end{tikzpicture}


%%% Local Variables: 
%%% mode: latex
%%% TeX-master: "../poster"
%%% End: 

      \begin{align*}
        \pots(\msx; \msc) &\triangleq  \tfrac{L'(1)}{R'(1)} \ent{R^{\cnop}(\msx)} + L'(1) \ent{ \rho^{\cnop}(\msx) } \\
        & \qquad - L'(1) \ent{\msx \cnop \rho^{\cnop}(\msx)} - \ent{ \msc \vnop L^{\vnop} \left( \rho^{\cnop}(\msx) \right) } 
      \end{align*}
      \vspace{1cm}
      \textcolor{jblue}{\bf Key Properties}
      \begin{itemize}
      \item Negative of the replica-symmetric free energy
      \item\vspace{0.75cm} Using duality rule
        \vspace{0.75cm}
        \begin{align*}
          \deri{\msx} \pots( \msx ; \msc ) [\msy] &= L'(1) \mathrm{H} \Big{(} \Big{[} \underbrace{\msc \vnop \lambda^{\vnop}(\rho^{\cnop}(\msx)) - \msx}_{\alert{\text{DE update}}} \Big{]} \cnop \left[ \rho'^{\cnop}(\msx) \cnop \msy \right] \Big{)}
        \end{align*}
      \item\vspace{0.75cm} A fixed point of DE is a \textcolor{red}{stationary} point of potential

      \end{itemize}
      \vspace{1cm}
      \textcolor{jblue}{\bf Energy Gap}
      \begin{itemize}
      \item\vspace{0.75cm} Defined by
        \begin{align*}
          \Delta E(\msc) = \inf_{\text{$\msx$ such that $\msx^{(\ell)}(\msx ; \msc) \not \to \cnunit$}} \pots(\msx ; \msc)
        \end{align*}
      \item\vspace{0.75cm} Monotonicity: $\Delta E(\msc(\h_1)) < \Delta E(\msc(\h_2)) $ when $\h_1 > \h_2$
      \item\vspace{0.75cm} Correspondence to MAP threshold:
        \begin{align*}
          \Delta E(\msc(\h)) > 0  \longleftrightarrow  \h < \hmap
        \end{align*}
      \end{itemize}
    \end{block}

    \vspace{2cm}
    \begin{block}{Spatially-Coupled DE}
      \begin{itemize}
      \item\vspace{0.75cm} DE update at check-node inputs
        \begin{align*}
          \label{equation:scde_update_1_ldpc}
          \msx_{i}^{(\ell+1)} = \frac{1}{w} \sum_{k=0}^{w-1} \msc_{i-k} \vnop \lambda^{\vnop} \left(  \frac{1}{w} \sum_{j=0}^{w-1} \rho^{\cnop} \left( \ms{x}_{i-k+j}^{(\ell)} \right) \right) ,
        \end{align*}
        where $\msc_i=\cnunit$ at the boundary and $\msc_i=\msc$ otherwise
      \item\vspace{0.75cm} \alert{Modified system} is a convenient upper bound on the original system.
        It is defined by
        \begin{align*}
          \msx_{i}^{(\ell)} = \msx_{i_0}^{(\ell)},  \quad \text{where $i_0$ is the mid-section}
        \end{align*}
      \item\vspace{0.75cm} Entropies of two systems
        \vspace{1cm}
        \begin{center}
          \resizebox{18cm}{9cm}{
            \begin{tikzpicture}[scale=0.2][domain=-20:20]
  \draw[very thin,color=gray] (-9,-1.25) -- (-9,1.25) ; 
  \draw[very thin,color=gray] (9,-1.25) -- (9,1.25) ; 
  \draw[very thin,color=gray] (0,-0.25) -- (0,0.25) node[below]{\tiny{$i_0$}}; 
  \draw[very thin,color=gray] (-16,0) -- (16,0) ; 

  \draw[color=red,very thick] 
  (-9,1.02746)--(-8.6,2.40972)--(-8.2,4.03658)--(-7.8,5.80014)--(-7.4,7.62622)--(-7,9.4749)--(-6.6,10.3015)--(-6.2,10.7752)--(-5.8,11.0051)--(-5.4,11.0985)--(-5,11.1294)--(-4.6,11.1378)--(-4.2,11.1407)--(-3.8,11.1418)--(-3.4,11.1421)--(-3,11.1422)--(-2.6,11.1423)--(-2.2,11.1423)--(-1.8,11.1423)--(-1.4,11.1423)--(-1,11.1423)--(-0.6,11.1423)--(-0.2,11.1423)--(0.2,11.1423)--(0.6,11.1423)--(1,11.1423)--(1.4,11.1423)--(1.8,11.1423)--(2.2,11.1423)--(2.6,11.1423)--(3,11.1423)--(3.4,11.1423)--(3.8,11.1423)--(4.2,11.1423)--(4.6,11.1423)--(5,11.1423)--(5.4,11.1423)--(5.8,11.1423)--(6.2,11.1423)--(6.6,11.1423)--(7,11.1423)--(7.4,11.1423)--(7.8,11.1423)--(8.2,11.1423)--(8.6,11.1423)--(9,11.1423);
  \draw[color=red,very thick,dashed] (9,0.25) -- (16,0.25);
  \draw[color=red,very thick,dashed] (-16,0.25) -- (-9,0.25);
  
  \draw[color=blue,very thick,yshift=-0.7cm] 
  (-9,1.02746)--(-8.6,2.40972)--(-8.2,4.03658)--(-7.8,5.80014)--(-7.4,7.62622)--(-7,9.4749)--(-6.6,10.3015)--(-6.2,10.7752)--(-5.8,11.0051)--(-5.4,11.0985)--(-5,11.1294)--(-4.6,11.1378)--(-4.2,11.1407)--(-3.8,11.1418)--(-3.4,11.1421)--(-3,11.1422)--(-2.6,11.1423)--(-2.2,11.1423)--(-1.8,11.1423)--(-1.4,11.1423)--(-1,11.1423)--(-0.6,11.1423)--(-0.2,11.1423)--(0.2,11.1423)--(0.6,11.1423)--(1,11.1423)--(1.4,11.1423)--(1.8,11.1423)--(2.2,11.1423)--(2.6,11.1423)--(3,11.1422)--(3.4,11.1421)--(3.8,11.1418)--(4.2,11.1407)--(4.6,11.1378)--(5,11.1294)--(5.4,11.0985)--(5.8,11.0051)--(6.2,10.7752)--(6.6,10.3015)--(7,9.4749)--(7.4,7.62622)--(7.8,5.80014)--(8.2,4.03658)--(8.6,2.40972)--(9,1.02746);
  \draw[color=blue,very thick] (9,0) -- (16,0);
  \draw[color=blue,very thick] (-16,0) -- (-9,0);

\end{tikzpicture}

%%% Local Variables: 
%%% mode: latex
%%% TeX-master: "../poster"
%%% End: 


          }
        \end{center}
      \item One can construct the \alert{coupled potential} $\potc$ so that fixed point of coupled DE is a stationary point
      \end{itemize}
    \end{block}

    \vspace{2cm}
    \begin{block}{Proof Outline (Achievability)}
      \begin{itemize}
        \vspace{0.4cm}
      \item For the \textcolor{blue}{modified system}, when $\Delta E(\msc)>0$, assume there is a non-trivial fixed point (i.e.~$\msbx \degr \underline{\Delta_{\infty}}$)
        \vspace{0.4cm}
      \item\vspace{0.75cm} Consider a small perturbation of the non-trivial fixed point
        \begin{itemize}
        \item\vspace{0.75cm} Choose \textcolor{blue}{right-shift} of the vector
        \item\vspace{0.75cm} Show the coupled potential \alert{decreases by a constant} along this perturbation
        \end{itemize}
      \item\vspace{0.75cm} Make all the variations in the potential up to second-order \textcolor{blue}{arbitrarily small} by choosing a \alert{large $w$}
      \item\vspace{0.75cm} These two observations \alert{contradict} each other
      \end{itemize}
    \end{block}
  \end{column}

  % Third Column
  \begin{column}{\threecolwid}

    \vspace{2cm}
    \begin{block}{Analysis of LDGM Codes}
      \vspace{0.75cm}
      \setlength\tikzheight{14.5cm}
      \setlength\tikzwidth{19cm}
      % This file was created by matlab2tikz v0.1.2.
% Copyright (c) 2008--2011, Nico Schlömer <nico.schloemer@gmail.com>
% All rights reserved.
% 
% The latest updates can be retrieved from
%   http://www.mathworks.com/matlabcentral/fileexchange/22022-matlab2tikz
% where you can also make suggestions and rate matlab2tikz.
% 

\begin{tikzpicture}

% defining custom colors
\definecolor{mycolor1}{rgb}{0,0.5,0}
\definecolor{mycolor2}{rgb}{0,0.75,0.75}
\definecolor{mycolor3}{rgb}{0.75,0,0.75}
\definecolor{mycolor4}{rgb}{0.75,0.75,0}
\definecolor{gray1}{rgb}{0.2,0.2,0.2}

\begin{axis}[%
scale only axis,
width=\tikzwidth,
height=\tikzheight,
xmin=0, xmax=1,
ymin=-0.2, ymax=0.6,
xlabel={\Large{$\tilde{\h}$}},
ylabel={\Large{$\pots \left( \mathrm{BAWGNC}(\tilde{\h}) ; \mathrm{BSC}(\h) \right)$}},
xtick={0,0.2,0.4,0.6,0.8,1},
ytick={-0.2,0,0.2,0.4,0.6},
xmajorgrids,
ymajorgrids,
zmajorgrids,
]

\addplot [
color=blue,
solid,
line width=3pt,
]
coordinates{
 (0,-0.000933955)(0.00204082,-0.000961297)(0.00408163,-0.000935065)(0.00612245,-0.000854929)(0.00816327,-0.000721009)(0.0102041,-0.000533563)(0.0122449,-0.000292971)(0.0142857,4.9838e-07)(0.0163265,0.000346246)(0.0183673,0.000743823)(0.0204082,0.00119293)(0.022449,0.00169298)(0.0244898,0.00224331)(0.0265306,0.00284385)(0.0285714,0.00349352)(0.0306122,0.00419198)(0.0326531,0.00493896)(0.0346939,0.00573352)(0.0367347,0.0065765)(0.0387755,0.00746499)(0.0408163,0.00839965)(0.0428571,0.00937991)(0.044898,0.0104046)(0.0469388,0.0114741)(0.0489796,0.0125869)(0.0510204,0.0137436)(0.0530612,0.0149423)(0.055102,0.0161826)(0.0571429,0.0174651)(0.0591837,0.0187879)(0.0612245,0.0201508)(0.0632653,0.021554)(0.0653061,0.0229957)(0.0673469,0.0244731)(0.0693878,0.0259913)(0.0714286,0.0275455)(0.0734694,0.0291375)(0.0755102,0.0307648)(0.077551,0.0324283)(0.0795918,0.0341258)(0.0816327,0.0358575)(0.0836735,0.0376239)(0.0857143,0.0394216)(0.0877551,0.0412533)(0.0897959,0.0431155)(0.0918367,0.045009)(0.0938776,0.0469336)(0.0959184,0.048888)(0.0979592,0.0508722)(0.1,0.0528835)(0.1,0.0528835)(0.118367,0.0721905)(0.136735,0.0933451)(0.155102,0.115928)(0.173469,0.13953)(0.191837,0.163771)(0.210204,0.188293)(0.228571,0.212766)(0.246939,0.236892)(0.265306,0.260411)(0.283673,0.283093)(0.302041,0.30475)(0.320408,0.325227)(0.338776,0.344408)(0.357143,0.362208)(0.37551,0.378576)(0.393878,0.393494)(0.412245,0.406968)(0.430612,0.419026)(0.44898,0.429723)(0.467347,0.439125)(0.485714,0.447315)(0.504082,0.454385)(0.522449,0.460441)(0.540816,0.46557)(0.559184,0.469894)(0.577551,0.473511)(0.595918,0.476522)(0.614286,0.479023)(0.632653,0.481103)(0.65102,0.482845)(0.669388,0.484323)(0.687755,0.485601)(0.706122,0.486737)(0.72449,0.487777)(0.742857,0.488762)(0.761224,0.489722)(0.779592,0.490681)(0.797959,0.491659)(0.816327,0.492667)(0.834694,0.493712)(0.853061,0.494802)(0.871429,0.495937)(0.889796,0.497118)(0.908163,0.498342)(0.926531,0.499607)(0.944898,0.500912)(0.963265,0.502255)(0.981633,0.503643)(1,0.505073) 
};

\addplot [
color=red,
solid,
line width=3pt,
]
coordinates{
 (0,-0.00366855)(0.00204082,-0.00379784)(0.00408163,-0.00388725)(0.00612245,-0.00393564)(0.00816327,-0.00394307)(0.0102041,-0.00390966)(0.0122449,-0.00383566)(0.0142857,-0.0037213)(0.0163265,-0.00356699)(0.0183673,-0.00337308)(0.0204082,-0.00313989)(0.022449,-0.00286784)(0.0244898,-0.00255745)(0.0265306,-0.00220888)(0.0285714,-0.00182288)(0.0306122,-0.00139981)(0.0326531,-0.000939921)(0.0346939,-0.000443932)(0.0367347,8.85751e-05)(0.0387755,0.000655656)(0.0408163,0.00125761)(0.0428571,0.00189395)(0.044898,0.00256381)(0.0469388,0.00326735)(0.0489796,0.00400345)(0.0510204,0.00477239)(0.0530612,0.0055729)(0.055102,0.00640445)(0.0571429,0.00726749)(0.0591837,0.00816056)(0.0612245,0.00908349)(0.0632653,0.0100363)(0.0653061,0.0110176)(0.0673469,0.0120254)(0.0693878,0.0130632)(0.0714286,0.0141277)(0.0734694,0.0152197)(0.0755102,0.0163377)(0.077551,0.0174822)(0.0795918,0.0186515)(0.0816327,0.0198456)(0.0836735,0.021065)(0.0857143,0.0223071)(0.0877551,0.0235737)(0.0897959,0.0248623)(0.0918367,0.0261734)(0.0938776,0.0275067)(0.0959184,0.0288614)(0.0979592,0.0302372)(0.1,0.0316324)(0.1,0.0316324)(0.118367,0.045034)(0.136735,0.0596988)(0.155102,0.0752916)(0.173469,0.0914921)(0.191837,0.108007)(0.210204,0.124565)(0.228571,0.140921)(0.246939,0.156864)(0.265306,0.172207)(0.283673,0.186796)(0.302041,0.200508)(0.320408,0.213248)(0.338776,0.22495)(0.357143,0.235574)(0.37551,0.245104)(0.393878,0.253548)(0.412245,0.260931)(0.430612,0.267296)(0.44898,0.2727)(0.467347,0.277211)(0.485714,0.280905)(0.504082,0.283863)(0.522449,0.286174)(0.540816,0.287917)(0.559184,0.289183)(0.577551,0.290053)(0.595918,0.290604)(0.614286,0.29091)(0.632653,0.291037)(0.65102,0.291043)(0.669388,0.290981)(0.687755,0.290894)(0.706122,0.290819)(0.72449,0.290785)(0.742857,0.290815)(0.761224,0.290924)(0.779592,0.291123)(0.797959,0.291418)(0.816327,0.291811)(0.834694,0.292299)(0.853061,0.292881)(0.871429,0.29355)(0.889796,0.294299)(0.908163,0.295122)(0.926531,0.296011)(0.944898,0.29696)(0.963265,0.297966)(0.981633,0.299032)(1,0.300154) 
};

\addplot [
color=mycolor1,
solid,
line width=3pt,
]
coordinates{
 (0,-0.0143502)(0.00204082,-0.0147289)(0.00408163,-0.0150847)(0.00612245,-0.0154146)(0.00816327,-0.0157185)(0.0102041,-0.0159963)(0.0122449,-0.016248)(0.0142857,-0.0164739)(0.0163265,-0.016674)(0.0183673,-0.0168487)(0.0204082,-0.0169981)(0.022449,-0.0171227)(0.0244898,-0.0172226)(0.0265306,-0.0172982)(0.0285714,-0.0173498)(0.0306122,-0.0173777)(0.0326531,-0.0173822)(0.0346939,-0.0173638)(0.0367347,-0.0173226)(0.0387755,-0.0172591)(0.0408163,-0.0171736)(0.0428571,-0.0170665)(0.044898,-0.0169382)(0.0469388,-0.016789)(0.0489796,-0.0166193)(0.0510204,-0.0164292)(0.0530612,-0.0162195)(0.055102,-0.0159904)(0.0571429,-0.015742)(0.0591837,-0.015475)(0.0612245,-0.0151897)(0.0632653,-0.0148862)(0.0653061,-0.0145652)(0.0673469,-0.0142276)(0.0693878,-0.0138723)(0.0714286,-0.0135008)(0.0734694,-0.0131129)(0.0755102,-0.0127093)(0.077551,-0.01229)(0.0795918,-0.0118559)(0.0816327,-0.011407)(0.0836735,-0.0109434)(0.0857143,-0.0104662)(0.0877551,-0.00997491)(0.0897959,-0.00947065)(0.0918367,-0.00895338)(0.0938776,-0.00842334)(0.0959184,-0.00788107)(0.0979592,-0.00732677)(0.1,-0.00676131)(0.1,-0.00676131)(0.118367,-0.001204)(0.136735,0.00502341)(0.155102,0.0116961)(0.173469,0.0186059)(0.191837,0.025567)(0.210204,0.032414)(0.228571,0.0390033)(0.246939,0.0452146)(0.265306,0.05095)(0.283673,0.0561322)(0.302041,0.0607068)(0.320408,0.0646378)(0.338776,0.0679085)(0.357143,0.0705181)(0.37551,0.0724813)(0.393878,0.0738256)(0.412245,0.0745893)(0.430612,0.0748195)(0.44898,0.0745704)(0.467347,0.0739008)(0.485714,0.0728726)(0.504082,0.0715492)(0.522449,0.0699918)(0.540816,0.0682661)(0.559184,0.0664258)(0.577551,0.0645269)(0.595918,0.0626187)(0.614286,0.0607451)(0.632653,0.0589445)(0.65102,0.0572493)(0.669388,0.055685)(0.687755,0.0542717)(0.706122,0.0530237)(0.72449,0.0519503)(0.742857,0.0510549)(0.761224,0.0503378)(0.779592,0.0497951)(0.797959,0.0494202)(0.816327,0.0492043)(0.834694,0.0491369)(0.853061,0.0492067)(0.871429,0.0494015)(0.889796,0.0497088)(0.908163,0.0501161)(0.926531,0.0506104)(0.944898,0.0511816)(0.963265,0.0518236)(0.981633,0.0525362)(1,0.0533136) 
};

\addplot [
color=mycolor2,
solid,
line width=3pt,
]
coordinates{
 (0,-0.0210384)(0.00204082,-0.0215335)(0.00408163,-0.0220115)(0.00612245,-0.0224684)(0.00816327,-0.0229039)(0.0102041,-0.023318)(0.0122449,-0.0237104)(0.0142857,-0.0240814)(0.0163265,-0.0244309)(0.0183673,-0.0247592)(0.0204082,-0.0250665)(0.022449,-0.025353)(0.0244898,-0.0256189)(0.0265306,-0.0258646)(0.0285714,-0.0260902)(0.0306122,-0.026296)(0.0326531,-0.0264824)(0.0346939,-0.0266496)(0.0367347,-0.026798)(0.0387755,-0.0269276)(0.0408163,-0.0270389)(0.0428571,-0.0271322)(0.044898,-0.0272078)(0.0469388,-0.027266)(0.0489796,-0.0273071)(0.0510204,-0.0273315)(0.0530612,-0.0273394)(0.055102,-0.0273312)(0.0571429,-0.0273071)(0.0591837,-0.0272675)(0.0612245,-0.0272128)(0.0632653,-0.0271431)(0.0653061,-0.0270589)(0.0673469,-0.0269606)(0.0693878,-0.0268482)(0.0714286,-0.0267223)(0.0734694,-0.026583)(0.0755102,-0.0264307)(0.077551,-0.0262657)(0.0795918,-0.0260884)(0.0816327,-0.025899)(0.0836735,-0.0256977)(0.0857143,-0.0254851)(0.0877551,-0.0252613)(0.0897959,-0.0250267)(0.0918367,-0.0247816)(0.0938776,-0.0245261)(0.0959184,-0.0242607)(0.0979592,-0.0239857)(0.1,-0.0237015)(0.1,-0.0237015)(0.118367,-0.0207693)(0.136735,-0.0173084)(0.155102,-0.0135093)(0.173469,-0.0095457)(0.191837,-0.00557156)(0.210204,-0.00172183)(0.228571,0.00188858)(0.246939,0.00516516)(0.265306,0.00803302)(0.283673,0.0104358)(0.302041,0.0123361)(0.320408,0.0137127)(0.338776,0.01456)(0.357143,0.0148863)(0.37551,0.0147118)(0.393878,0.0140673)(0.412245,0.012992)(0.430612,0.0115322)(0.44898,0.00973844)(0.467347,0.00766526)(0.485714,0.00536848)(0.504082,0.00290437)(0.522449,0.000325336)(0.540816,-0.00230859)(0.559184,-0.00495522)(0.577551,-0.00756664)(0.595918,-0.0101025)(0.614286,-0.0125274)(0.632653,-0.0148116)(0.65102,-0.0169305)(0.669388,-0.0188663)(0.687755,-0.0206057)(0.706122,-0.0221406)(0.72449,-0.0234675)(0.742857,-0.0245882)(0.761224,-0.0255067)(0.779592,-0.0262308)(0.797959,-0.0267704)(0.816327,-0.0271375)(0.834694,-0.0273445)(0.853061,-0.0274052)(0.871429,-0.0273333)(0.889796,-0.0271428)(0.908163,-0.0268477)(0.926531,-0.0264617)(0.944898,-0.0259962)(0.963265,-0.0254575)(0.981633,-0.0248461)(1,-0.024169) 
};

\addplot [
color=mycolor3,
solid,
line width=3pt,
]
coordinates{
 (0,-0.0300511)(0.00204082,-0.0306788)(0.00408163,-0.0312947)(0.00612245,-0.0318938)(0.00816327,-0.0324757)(0.0102041,-0.0330401)(0.0122449,-0.0335869)(0.0142857,-0.0341163)(0.0163265,-0.0346279)(0.0183673,-0.0351221)(0.0204082,-0.0355991)(0.022449,-0.0360591)(0.0244898,-0.036502)(0.0265306,-0.0369284)(0.0285714,-0.0373382)(0.0306122,-0.0377317)(0.0326531,-0.0381094)(0.0346939,-0.0384711)(0.0367347,-0.0388179)(0.0387755,-0.0391489)(0.0408163,-0.0394649)(0.0428571,-0.0397662)(0.044898,-0.0400528)(0.0469388,-0.0403254)(0.0489796,-0.0405839)(0.0510204,-0.0408289)(0.0530612,-0.0410603)(0.055102,-0.0412784)(0.0571429,-0.0414839)(0.0591837,-0.0416766)(0.0612245,-0.041857)(0.0632653,-0.0420253)(0.0653061,-0.0421818)(0.0673469,-0.0423265)(0.0693878,-0.0424602)(0.0714286,-0.0425827)(0.0734694,-0.0426947)(0.0755102,-0.042796)(0.077551,-0.0428873)(0.0795918,-0.0429685)(0.0816327,-0.04304)(0.0836735,-0.0431021)(0.0857143,-0.0431549)(0.0877551,-0.0431989)(0.0897959,-0.0432342)(0.0918367,-0.0432611)(0.0938776,-0.0432798)(0.0959184,-0.0432906)(0.0979592,-0.0432937)(0.1,-0.0432894)(0.1,-0.0432894)(0.118367,-0.042957)(0.136735,-0.0422139)(0.155102,-0.041217)(0.173469,-0.040107)(0.191837,-0.0390075)(0.210204,-0.0380247)(0.228571,-0.0372471)(0.246939,-0.0367457)(0.265306,-0.0365747)(0.283673,-0.0367721)(0.302041,-0.0373607)(0.320408,-0.0383496)(0.338776,-0.0397351)(0.357143,-0.0415024)(0.37551,-0.043627)(0.393878,-0.0460767)(0.412245,-0.0488129)(0.430612,-0.0517915)(0.44898,-0.054966)(0.467347,-0.0582873)(0.485714,-0.0617062)(0.504082,-0.0651738)(0.522449,-0.0686466)(0.540816,-0.0720706)(0.559184,-0.0754149)(0.577551,-0.0786393)(0.595918,-0.0817121)(0.614286,-0.0846065)(0.632653,-0.0873004)(0.65102,-0.0897767)(0.669388,-0.0920247)(0.687755,-0.0940373)(0.706122,-0.0958121)(0.72449,-0.0973509)(0.742857,-0.0986597)(0.761224,-0.0997466)(0.779592,-0.100623)(0.797959,-0.101301)(0.816327,-0.101796)(0.834694,-0.102122)(0.853061,-0.102295)(0.871429,-0.102329)(0.889796,-0.102241)(0.908163,-0.102045)(0.926531,-0.101756)(0.944898,-0.101385)(0.963265,-0.10094)(0.981633,-0.100422)(1,-0.0998381) 
};

\addplot [
color=mycolor4,
solid,
line width=3pt,
]
coordinates{
 (0,-0.0438963)(0.00204082,-0.0446936)(0.00408163,-0.0454851)(0.00612245,-0.0462643)(0.00816327,-0.0470308)(0.0102041,-0.0477841)(0.0122449,-0.0485238)(0.0142857,-0.0492503)(0.0163265,-0.0499631)(0.0183673,-0.0506622)(0.0204082,-0.0513481)(0.022449,-0.0520207)(0.0244898,-0.0526799)(0.0265306,-0.0533265)(0.0285714,-0.0539599)(0.0306122,-0.0545806)(0.0326531,-0.055189)(0.0346939,-0.055785)(0.0367347,-0.0563697)(0.0387755,-0.0569416)(0.0408163,-0.0575019)(0.0428571,-0.0580507)(0.044898,-0.058588)(0.0469388,-0.0591144)(0.0489796,-0.0596297)(0.0510204,-0.0601347)(0.0530612,-0.060629)(0.055102,-0.0611128)(0.0571429,-0.061587)(0.0591837,-0.0620511)(0.0612245,-0.0625056)(0.0632653,-0.062951)(0.0653061,-0.063387)(0.0673469,-0.0638133)(0.0693878,-0.0642318)(0.0714286,-0.0646414)(0.0734694,-0.065043)(0.0755102,-0.0654362)(0.077551,-0.0658219)(0.0795918,-0.0661996)(0.0816327,-0.0665699)(0.0836735,-0.0669331)(0.0857143,-0.0672889)(0.0877551,-0.0676382)(0.0897959,-0.0679805)(0.0918367,-0.0683165)(0.0938776,-0.0686462)(0.0959184,-0.0689699)(0.0979592,-0.0692878)(0.1,-0.0695997)(0.1,-0.0695997)(0.118367,-0.0721848)(0.136735,-0.0744588)(0.155102,-0.0765428)(0.173469,-0.0785427)(0.191837,-0.0805509)(0.210204,-0.0826441)(0.228571,-0.0848842)(0.246939,-0.0873191)(0.265306,-0.0899828)(0.283673,-0.0928958)(0.302041,-0.0960675)(0.320408,-0.099496)(0.338776,-0.10317)(0.357143,-0.10707)(0.37551,-0.111168)(0.393878,-0.115433)(0.412245,-0.119829)(0.430612,-0.124314)(0.44898,-0.128849)(0.467347,-0.133391)(0.485714,-0.137899)(0.504082,-0.142333)(0.522449,-0.14666)(0.540816,-0.150832)(0.559184,-0.15483)(0.577551,-0.158625)(0.595918,-0.162192)(0.614286,-0.165514)(0.632653,-0.168579)(0.65102,-0.171375)(0.669388,-0.173901)(0.687755,-0.176154)(0.706122,-0.17814)(0.72449,-0.179865)(0.742857,-0.18134)(0.761224,-0.182576)(0.779592,-0.183588)(0.797959,-0.184392)(0.816327,-0.185004)(0.834694,-0.18544)(0.853061,-0.185719)(0.871429,-0.185856)(0.889796,-0.185867)(0.908163,-0.185769)(0.926531,-0.185576)(0.944898,-0.185302)(0.963265,-0.184954)(0.981633,-0.184533)(1,-0.184047) 
};

\addplot [
color=black,
only marks,
mark=star,
mark options={solid},
mark size=10pt
]
coordinates{
 (0.00204082,-0.000961297)(0.00816327,-0.00394307)(0.0326531,-0.0173822)(0.0530612,-0.0273394)(0.0979592,-0.0432937)(0.889796,-0.1859) 
};

\addplot [
color=gray1,
thick,
dashed,
]
coordinates{
 (0,-0.0173822)(1,-0.0173822)
};

\addplot[<->,thick,color=gray1]  coordinates{(0.834694,0.0491369)(0.834694,-0.0173822)} node [pos=0.75,pin={120:\large{$\Delta E$}},inner sep=0pt] {};

\end{axis}
\end{tikzpicture}


%%% Local Variables: 
%%% mode: latex
%%% TeX-master: "../poster"
%%% End: 


      \vspace{1.25cm}
      \textcolor{jblue}{\bf Potential Functional}

      \begin{itemize}
      \item\vspace{0.75cm} Same key properties
      \item\vspace{0.75cm} However, main issue is that $\cnunit$ is not a fixed point of DE or a stationary point of potential
      \item\vspace{0.75cm} Energy gap \alert{may not} be monotonic (difficulty in defining thresholds)
      \end{itemize}

      \vspace{1.25cm}
      \textcolor{jblue}{\bf Proof Strategy}
      \begin{itemize}
      \item\vspace{0.75cm} For the SC system, change the boundary to \alert{minimal fixed point}
      \item\vspace{0.75cm} Tweak the coupled potential to reflect the boundary and show that it has the \alert{desired} properties
      \end{itemize}
    \end{block}

    \setbeamercolor{block alerted title}{fg=black,bg=norange} % Change the alert block title colors
    \setbeamercolor{block alerted body}{fg=black,bg=white} % Change the alert block body colors

    \vspace{2.5cm}
    \begin{alertblock}{Conclusion}
      \begin{itemize}
      \item Proved \textcolor{blue}{threshold saturation} for irregular LDPC and LDGM codes on BMS channels
      \item\vspace{0.75cm} Simple proof based on \textcolor{blue}{potential functions}
      \item\vspace{0.75cm} Can be expanded to other Graphical models
        \begin{itemize}
        \item\vspace{0.75cm} Can handle \alert{non-trivial fixed points}
        \end{itemize}
      \item\vspace{0.75cm} Interesting to see if one can analyze SC LDGM codes with belief propagation guided decimation for lossy source compression 
      \end{itemize}
    \end{alertblock}

    % Bibliography
    \vspace{2.5cm}
    \begin{block}{References}

      \begin{thebibliography}{10}
        \vspace{1cm}
      \bibitem{Felstrom-it99}
        J.~Felstrom and K.~S. Zigangirov, ``Time-varying periodic convolutional codes
        with low-density parity-check matrix,'' \emph{IEEE Trans.\ Inform.\ Theory},
        vol.~45, no.~6, pp. 2181--2191, 1999.
        \vspace{1cm}
      \bibitem{Lentmaier-it10}
        M.~Lentmaier, A.~Sridharan, D.~J. Costello, and K.~S. Zigangirov, ``Iterative
        decoding threshold analysis for {LDPC} convolutional codes,'' \emph{IEEE
          Trans.\ Inform.\ Theory}, vol.~56, no.~10, pp. 5274--5289, Oct. 2010.
        \vspace{1cm}
      \bibitem{Kudekar-it11}
        S.~Kudekar, T.~J. Richardson, and R.~L. Urbanke, ``Threshold saturation via
        spatial coupling: {W}hy convolutional {LDPC} ensembles perform so well over
        the {BEC},'' \emph{IEEE Trans.\ Inform.\ Theory}, vol.~57, no.~2, pp.
        803--834, 2011.
        \vspace{1cm}
      \bibitem{Kudekar-arxiv12}
        S.~Kudekar, T.~Richardson, and R.~Urbanke, ``Spatially coupled ensembles
        universally achieve capacity under belief propagation,'' 2012, arxiv preprint,
        arXiv:1201.2999.
        \vspace{1cm}
      \bibitem{Yedla-istc12}
        A.~Yedla, Y.-Y. Jian, P.~S. Nguyen, and H.~D. Pfister, ``A simple proof of
        threshold saturation for coupled scalar recursions,'' in \emph{Proc.\ Int.\
          Symp.\ on Turbo Codes \& Iterative Inform.\ Proc.}, 2012, pp. 51--55, arxiv
        preprint arXiv:1204.5703, 2012.
        \vspace{1cm}
      \bibitem{Montanari-it05}
        A.~Montanari, ``Tight bounds for {LDPC} and {LDGM} codes under {MAP}
        decoding,'' \emph{IEEE Trans.\ Inform.\ Theory}, vol.~51, no.~9, pp.
        3221--3246, 2005.
      \end{thebibliography}
    \end{block}
  \end{column}
\end{columns}
\end{document}
